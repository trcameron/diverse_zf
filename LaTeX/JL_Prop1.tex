\documentclass{article}
\usepackage{amsmath}
\usepackage{amsthm}
\usepackage{amssymb}
\usepackage{dirtytalk}
\usepackage{enumerate}
\usepackage[margin=1.00in]{geometry}
\usepackage{graphicx}
\usepackage{hyperref}
\usepackage{optidef}
\usepackage{tikz}






\setlength{\headheight}{-.25in}
\setlength{\topmargin}{-.25in}
\setlength{\textwidth}{6.5in}
\setlength{\textheight}{9.6in}
\setlength{\oddsidemargin}{0in}
\setlength{\evensidemargin}{\oddsidemargin}

\newcommand{\ds}{\displaystyle}
\renewcommand{\L}{\mathcal{L}}
\newtheorem{qu}{}
\newenvironment{ques}{\hspace{-.25in} \begin{qu} \rm}{\end{qu}}



% new commands and math operators
\newcommand\abs[1]{\left|#1\right|}
\newcommand\zip[1]{\operatorname{ZIP}\left(#1\right)}
\newcommand\dzf[1]{\operatorname{DZF}\left(#1\right)}

% new theorems and definitions
\newtheorem{theorem}{Theorem}[section]
\newtheorem{proposition}[theorem]{Proposition}
\newtheorem{corollary}[theorem]{Corollary}
\newtheorem{lemma}[theorem]{Lemma}
\newtheorem{question}[theorem]{Question}
\theoremstyle{definition}
\newtheorem{remark}[theorem]{Remark}

\begin{document}
\noindent
{MATH 494}
\hfill
{
Jakob Loedding}

\vspace{.3 in}

\centerline{\large \bf  L{\small INEAR} P{\small ROGRAMMING,} Z{\small ERO-}F{\small ORCING,} {\small AND} M{\small AXIMALLY} D{\small IVERSE} O{\small PTIMA} }
\vspace{.1 in}
\centerline{\large  (P{\small ROGRESS} D{\small OCUMENTATION})}

\vspace{.2 in}

\section{Diversity Model/Integer Program}\label{sec:intro}
Denote the diversity model as $\dzf{G,s^{*}}$, where $G \in \mathbb{G}$ and $s^{*}$ is an optimal solution to $\zip{G}$.

%%%%%%%%%%%%%%%%%%%%%
%    Diverse Zero Forcing IP Model        %
%%%%%%%%%%%%%%%%%%%%%
\begin{mini!}
	{}{\sum_{v\in V}z_{v}}{}{}\label{eq:zfip-obj}
	\addConstraint{s_{v}+\sum_{e\in\delta^{-}(v)}y_{e}}{=1,~\quad\forall \ v\in V}\label{eq:zfip-const1}
	\addConstraint{s_{v}^{'}+\sum_{e\in\delta^{-}(v)}y_{e}^{'}}{=1,~\quad\forall \ v\in V}\label{eq:zfip-const2}
	\addConstraint{x_{u}-x_{v}+(T+1)y_{e}}{\leq T,~\quad\forall \ e=(u,v)\in E}\label{eq:zfip-const3}
	\addConstraint{x_{u}^{'}-x_{v}^{'}+(T+1)y_{e}^{'}}{\leq T,~\quad\forall \ e=(u,v)\in E}\label{eq:zfip-const4}
	\addConstraint{x_{w}-x_{v}+(T+1)y_{e}}{\leq T,~\quad\forall \ e=(u,v)\in E,~\forall \ w\in N(u)\setminus\{v\}}\label{eq:zfip-const5}
	\addConstraint{x_{w}^{'}-x_{v}^{'}+(T+1)y_{e}^{'}}{\leq T,~\quad\forall \ e=(u,v)\in E,~\forall \ w\in N(u)\setminus\{v\}}\label{eq:zfip-const6}
	\addConstraint{\sum_{v \in V}s_{v}}{=s^{*}}\label{eq:zfip-const7}
	\addConstraint{\sum_{v \in V}s_{v}^{'}}{= s^{*}}\label{eq:zfip-const8}
	\addConstraint{s_{v}+s_{v}^{'}-z_{v}}{\leq T,~\quad\forall \ v \in V }\label{eq:zfip-const9}
	\addConstraint{s, s^{'}\in\{0,1\}^{n}, ~x, x^{'}\in\{0,\ldots,T\}^{n},~y, y^{'}\in\{0,1\}^{m}}{}\label{eq:zfip-const10}
\end{mini!} 

\begin{proposition}
Given an optimal solution to $\dzf{G,s^{*}}$, $z_v = 1$ for some $v \in V$ if and only if $s_v = s_v^{'} = 1$.
\end{proposition}

\begin{proof}
Suppose $z_v = 1$ for some $v \in V$.
By way of contradiction, also suppose either $s_v = 0$ or $s_v^{'} = 0$.
Then constraint  (\ref{eq:zfip-const9}) will be true if $z_v = 0$.
This contradicts the fact that we have an optimal solution to $\dzf{G,s^{*}}$, i.e., the objective function (\ref{eq:zfip-obj}) would not be minimized.
Thus, our assumption that $s_v = 0$ or $s_v^{'} = 0$ must be false.
Hence, $s_v = s_v^{'} = 1$.

Conversely, let $s_v = s_v^{'} = 1$ for some $v \in V$.
Then constraint (\ref{eq:zfip-const9}) holds only when $z_v = 1$. 
Therefore, given an optimal solution to $\dzf{G,s^{*}}$, $z_v = 1$ for some $v \in V$ if and only if $s_v = s_v^{'} = 1$.
\end{proof}

\end{document}
