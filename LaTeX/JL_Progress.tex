\documentclass{article}
\usepackage{amsmath}
\usepackage{amsthm}
\usepackage{amssymb}
\usepackage{dirtytalk}
\usepackage{enumerate}
\usepackage[margin=1.00in]{geometry}
\usepackage{graphicx}
\usepackage{hyperref}
\usepackage{optidef}
\usepackage{tikz}

\usepackage{mathtools}
\DeclarePairedDelimiter\floor{\lfloor}{\rfloor}

\setlength{\headheight}{-.25in}
\setlength{\topmargin}{-.25in}
\setlength{\textwidth}{6.5in}
\setlength{\textheight}{9.6in}
\setlength{\oddsidemargin}{0in}
\setlength{\evensidemargin}{\oddsidemargin}

\newcommand{\ds}{\displaystyle}
\renewcommand{\L}{\mathcal{L}}
\newtheorem{qu}{}
\newenvironment{ques}{\hspace{-.25in} \begin{qu} \rm}{\end{qu}}



% new commands and math operators
\newcommand\abs[1]{\left|#1\right|}
\newcommand\zip[1]{\operatorname{ZIP}\left(#1\right)}
\newcommand\zfd[1]{\operatorname{ZFD}\left(#1\right)}

% new theorems and definitions
\newtheorem{theorem}{Theorem}[section]
\newtheorem{proposition}[theorem]{Proposition}
\newtheorem{corollary}[theorem]{Corollary}
\newtheorem{lemma}[theorem]{Lemma}
\newtheorem{question}[theorem]{Question}
\theoremstyle{definition}
\newtheorem{remark}[theorem]{Remark}

\begin{document}
\noindent
{MATH 494}
\hfill
{
Jakob Loedding}

\vspace{.3 in}

\centerline{\large \bf  L{\small INEAR} P{\small ROGRAMMING,} Z{\small ERO-}F{\small ORCING,} {\small AND} M{\small AXIMALLY} D{\small IVERSE} O{\small PTIMA} }
\vspace{.1 in}
\centerline{\large  (P{\small ROGRESS} D{\small OCUMENTATION})}

\vspace{.2 in}

\section{Integer Program for Zero Forcing Diameter}\label{sec:intprog}
Denote the zero forcing diameter model as $\zfd{G,s^{*}}$, where $G \in \mathbb{G}$ and $s^{*}$ is an optimal solution to $\zip{G}$.
That is, $s^{*}$ is the zero forcing number of $G$.
The integer program $\zfd{G,s^{*}}$ presented in (\ref{eq:zfip-obj}) -(\ref{eq:zfip-const10}) computes the minimum intersection of two distinct zero forcing sets of $G$. We reference these sets by  $B$ and $B^{'}$.
Note that every edge in $E(G)$ is represented by two directed edges with opposite direction.
The binary variables $s_v$ and $s_v^{'}$ indicate whether $v$ is in $B$ or $B^{'}$ respectively.
The variables $x_v, x_v^{'} \in \mathbb{Z}$ ranging in \{0,1,...,$T$\} specify at which time steps vertex $v$ is forced given its corresponding initial zero forcing set.
Note that $T$ is the maximum difference in the forcing times of two vertices. 
The binary variables $y_e, y_e^{'}$,  for each directed edge $e = (u,v)$, indicate whether $u$ forces $v$ given $B$ or $B^{'}$ initially.
Lastly, the notation $\delta^{-}(v)$ references the set of edges pointing toward vertex $v$ and $N(u)$ denotes the set of all vertices $v \in V(G)$ that are adjacent to $u$. 
Note that the vertex and edge sets of $G$ are denoted by $V$ and $E$ respectively.

%%%%%%%%%%%%%%%%%%%%%
%    Zero Forcing  Diameter IP Model      %
%%%%%%%%%%%%%%%%%%%%%
\begin{mini!}
	{}{\sum_{v\in V}z_{v}}{}{}\label{eq:zfip-obj}
	\addConstraint{s_{v}+\sum_{e\in\delta^{-}(v)}y_{e}}{=1,~\quad\forall \ v\in V}\label{eq:zfip-const1}
	\addConstraint{s_{v}^{'}+\sum_{e\in\delta^{-}(v)}y_{e}^{'}}{=1,~\quad\forall \ v\in V}\label{eq:zfip-const2}
	\addConstraint{x_{u}-x_{v}+(T+1)y_{e}}{\leq T,~\quad\forall \ e=(u,v)\in E}\label{eq:zfip-const3}
	\addConstraint{x_{u}^{'}-x_{v}^{'}+(T+1)y_{e}^{'}}{\leq T,~\quad\forall \ e=(u,v)\in E}\label{eq:zfip-const4}
	\addConstraint{x_{w}-x_{v}+(T+1)y_{e}}{\leq T,~\quad\forall \ e=(u,v)\in E,~\forall \ w\in N(u)\setminus\{v\}}\label{eq:zfip-const5}
	\addConstraint{x_{w}^{'}-x_{v}^{'}+(T+1)y_{e}^{'}}{\leq T,~\quad\forall \ e=(u,v)\in E,~\forall \ w\in N(u)\setminus\{v\}}\label{eq:zfip-const6}
	\addConstraint{\sum_{v \in V}s_{v}}{=s^{*}}\label{eq:zfip-const7}
	\addConstraint{\sum_{v \in V}s_{v}^{'}}{= s^{*}}\label{eq:zfip-const8}
	\addConstraint{s_{v}+s_{v}^{'}-z_{v}}{\leq 1~\quad\forall \ v \in V }\label{eq:zfip-const9}
	\addConstraint{s, s^{'},z†\in\{0,1\}^{n}, ~x, x^{'}\in\{0,\ldots,T\}^{n},~y, y^{'}\in\{0,1\}^{m}}{}\label{eq:zfip-const10}
\end{mini!} 
The triples $(s,x,y)$ and $(s^{'},x^{'},y^{'})$, where $s,s^{'} \in \{0,1\}^{n}$; $x,x^{'} \in \{0,...,T\}^{n}$, and $y,y^{'} \in \{0,1\}^{m}$, make a  \textit{feasible solution} of $\zfd{G,s^{*}}$ provided that $(s,x,y)$ and $(s^{'},x^{'},y^{'})$ satisfy (\ref{eq:zfip-const1})-(\ref{eq:zfip-const10}).
Additionally, if $s$ and $s^{'}$ are minimal with respect to the objective function (\ref{eq:zfip-obj}), then we say that $(s,x,y)$ and $(s^{'},x^{'},y^{'})$ is an \textit{optimal solution} of $\zfd{G,s^{*}}$.
The corresponding minimal value of the objective function (\ref{eq:zfip-obj}) is the \textit{optimal value} of $\zfd{G,s^{*}}$.

\begin{proposition}
Given an optimal solution to $\zfd{G,s^{*}}$, $z_v = 1$ for some $v \in V$ if and only if $s_v = s_v^{'} = 1$.
\end{proposition}

\begin{proof}
Suppose $z_v = 1$ for some $v \in V$.
By way of contradiction, also suppose either $s_v = 0$ or $s_v^{'} = 0$.
Then constraint  (\ref{eq:zfip-const9}) will be true if $z_v = 0$.
This contradicts the fact that we have an optimal solution to $\zfd{G,s^{*}}$, i.e., the objective function (\ref{eq:zfip-obj}) would not be minimized.
Thus, our assumption that $s_v = 0$ or $s_v^{'} = 0$ must be false.
Hence, $s_v = s_v^{'} = 1$.

Conversely, let $s_v = s_v^{'} = 1$ for some $v \in V$.
Then constraint (\ref{eq:zfip-const9}) holds only when $z_v = 1$. 
Therefore, given an optimal solution to $\zfd{G,s^{*}}$, $z_v = 1$ for some $v \in V$ if and only if $s_v = s_v^{'} = 1$.
\end{proof}

\section{Zero Forcing Diameter/Graphical Interpretation of $\zfd{G,s^{*}}$}\label{sec:zfdiam}
 Given a \textit{feasible solution} of $\zfd{G,s^{*}}$, the value of the objective function (\ref{eq:zfip-obj}) can be defined as $|B \cap B^{'}|$, where $B$ and $B^{'}$ are zero forcing sets of $G$.
Furthermore, the \textit{optimal value} of $\zfd{G,s^{*}}$ is equal to $\text{min}|B \cap B^{'}|$, where $B,B^{'} \in S$ and $S$ is the set of all minimal zero forcing sets of $G$.
We define the \textit{zero forcing diameter} of a graph $G \in \mathbb{G}$ as 
\begin{align}
d(G) = zf(G) - \text{min}|B \cap B^{'}|
\end{align}
 where $B,B^{'} \in S$ and $zf(G)$ is the zero forcing number of $G$.
Note that $zf(G) = s^{*}$ in terms of the inter program $\zfd{G,s^{*}}$ given in equations (\ref{eq:zfip-obj}) -(\ref{eq:zfip-const10}).

\begin{proposition}
Given a graph $G \in \mathbb{G}$, $d(G) = 0$ if and only if $G$ is an empty graph.
\end{proposition}
\begin{proof}
Suppose $G \in \mathbb{G}$ and $d(G) = 0$, i.e., $zf(G) = \text{min}|B \cap B^{'}|$.
This implies there is only one unique zero-forcing set of $G$, which is only possible if no forcings occur.
Hence, our zero-forcing set must include all vertices in $G$.
This implies that $G$ has no edges.
Thus, $G$ is an empty graph by definition.

Conversely, suppose $G$ is an empty graph.
Since $G$ has no edges by definition, then the only way to force all of its vertices blue is to include them in our zero-forcing set.
Hence, $\text{min}|B \cap B^{'}| = zf(G)$ where $B$, $B^{'}$ are minimal zero-forcing sets of $G$.
Thus, $d(G) = 0$.
Therefore, $d(G) = 0$ if and only if $G$ is an empty graph.
\end{proof}

\begin{proposition}
Given a path graph $P_n$, $d(P_n)=1 \ \forall \ n \in \mathbb{N}$.
\end{proposition}
\begin{proof}
Suppose $P_n$ is a path graph such that $n  \in \mathbb{N}$.
By definition, $P_n$ contains at least 2 vertices and $zf(P_n) = 1$ [citation].
Thus, $\text{min}|B\cap B^{'}| = 0$, where $B$ and $B^{'}$ are minimum zero-forcing sets containing the leftmost and rightmost vertices of $P_n$ respectively. 
Therefore, $d(G) = 1$ by definition.
\end{proof}

\begin{proposition}
Given a complete graph $K_n$, $d(K_n)=1 \ \forall \ n \in \mathbb{N}$.
\end{proposition}
\begin{proof}
Suppose $K_n$ is a complete graph such that $n \in \mathbb{N}$.
Note that $zf(K_n) = n - 1$ [citation].
To calculate $d(K_n)$, let $B$ be a minimum zero-forcing set containing $n-1$ arbitrary vertices in $V(K_n)$.
Let $B^{'}$ be another minimum zero-forcing set containing the vertex in the set $V(K_n) - B$ in addition  to $n-2$ unique, arbitrary vertices in $V(K_n)$.
Since $|B| = n- 1$ and $V(K_n) - B \subseteq B^{'}$, then $\text{min}|B \cap B^{'}| = n-2$.
Therefore, $d(K_n) = (n-1) - (n-2) = 1$.
\end{proof}

\begin{proposition}
Given a star graph $S_n$, $d(S_n)=1 \ \forall \ n \in \mathbb{N}$.
\end{proposition}
\begin{proof}
Suppose $S_n$ is a star graph such that $n \in \mathbb{N}$ and let $L(S_n) \subseteq V(S_n)$ be the set of leaves in the graph $S_n$.
Note that $zf(S_n) = n - 2$ [citation].
To calculate $d(S_n)$, let $B$ be a minimum zero-forcing set of $S_n$ containing $|L(S_n)| - 1$ arbitrary leaves in $L(S_n)$, i.e., $n - 2$ vertices in $V(S_n)$.
Similarly, let $B^{'}$ be another minimum zero-forcing set of $S_n$ containing the vertex in the set $L(S_n) - B$ and  $|L(S_n)|-2$ unique, arbitrary leaves in $L(S_n)$.
Since $|B| = n - 2$, $L(S_n) - B \subseteq B^{'}$, and the root of $S_n$ is not in $B$ or $B^{'}$, then $\text{min}|B \cap B^{'}| = n - 3$.
Therefore, $d(S_n) = (n - 2) - (n-3) = 1$.
\end{proof}

\begin{proposition}
Given a cycle graph $C_n$, $d(C_n)=2 \ \forall \ n \geq 4 \in \mathbb{N}$.
\end{proposition}

\begin{theorem}
For all graphs $G \in \mathbb{G}$, we have that $0 \leq d(G) \leq \floor{n/2}$ where $n$ is the order of $G$.
\end{theorem}
\begin{proof}
Suppose $G \in \mathbb{G}$.
By definition, $0 \leq d(G) \leq zf(G)$.
Hence, $d(G) \geq 0$.
By way of contradiction, suppose $d(G) > \floor{n/2}$, i.e., $d(G) = \floor{n/2} + c$ where $0 < c < \floor{n/2}$.
Then $zf(G) - \text{min}|B \cap B^{'}| = \floor{n/2} + c$.
This implies $zf(G) \geq \floor{n/2} + c$, and it follows that  $\text{min}|B\cap B^{'}| \geq c$ for all zero-forcing sets B and B'. 
Hence, $d(G) \leq \floor{n/2}$, which contradicts our assumption that $d(G) > \floor{n/2}$.
Therefore, $0 \leq d(G) \leq \floor{n/2}$  for all $G \in \mathbb{G}$.
\end{proof}







\end{document}
